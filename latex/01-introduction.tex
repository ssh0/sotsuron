\chapter{導入}

私達は普段の身近な生活の中で、サイズが変わるとその中の質や状態が変わってしまうと感じるものを見つけることができる。
それは会話とその人数の間の関係にもあるかもしれないし、共同作業を行うときのチームメンバーの人数かもしれない。企業の大きさとその中の人間の従事度も異なるかもしれない。会議にやたら多くの人が参加していても、ほとんどの人はその会議を無意味なものだと見なすことになるだろう。世の中のおよそほとんどのスポーツやゲームなどは、参加する人数が予め決められていることが多い。よく練られたものであれば、この決められた人数より多くても少なくても、本来の楽しみを得ることはできないだろう。麻雀は4人でするから楽しめるのであり、会議は100人ではなかなか進まないものである。

このような現象はこれまで見てきたように大変身近で、私達にとって実感をもって受け入れられることなので、ほとんどの場合不文律にこの法則は受け入れられていると言って良いだろう。

また、相乗効果という点に着目すれば、一人で何かするよりも、多くの人を交えて行ったほうが効率が上がるといったものも多く存在していることだろう。しかしながら、そこでも多すぎはまた別の問題を産むなどして、多すぎることも効率を下げる要因になることもあるのである。

こういったことを私達は無自覚の内に理解しているので、何かを大人数で共同して行うときには、それをさらにいくつかのグループ、班、部署、そういったより小さい単位に分割する必要を感じる。もちろん、このような現象はよく知られたことであるので、どれほどのシステムサイズであれば最大の効率が導き出せるか、ということは、昔から考えるべきことであり、最適人数に関する調査・研究が行われたり、詳細な研究はなされていなくても各分野での情報が蓄積されたりしている。例えば教育の分野などではグループ学習や少人数授業の有効性が議論されることも多い。

さて、これまでに述べたような系のシステムサイズとその効率の間の関係は、何も人間を要素とした系でなくとも考えることが出来る。例えば、動物の体重と代謝率の間の関係も、同じ現象であるとみなすことができるかもしれない。動物学等の分野で知られている法則として、体重とその他の観測量の間の関係としてアロメトリー則と呼ばれるものがある。その一つが体重と代謝率の間の関係であり、特にこの関係のことをKleiber則と呼ぶ。これは様々なスケールの対象について調べられていて、代謝率$E$と体重$M$の間の関係は$E\sim M^{b}$のようにスケールされる。この指数については、データの取り方などによって異なるため、いくつか説があり、指数の大きさは$2/3$とする意見と$3/4$とする意見とがあり、また両辺の対数を取った関係式で右辺の$\log M$の2乗の項を考慮すればよりよくフィッティングできる、といったものもある。ここで重要視したいのは、指数の大きさがどちらか、ということではなく、その指数は1よりも小さい値をとる、ということである。これまで挙げてきたシステムサイズと効率の関係についてと同じように議論する余地があることが分かる。

この指数が生物種や成長過程における細胞組成の変化が本質的ではなく、その時点でのシステムのサイズが影響していることの一つの例として、ホヤの群体サイズと代謝率の間の関係も、同様に1乗より小さい指数でスケールされることが確かめられている。このホヤは無性生殖によって自分と全く同じ個体を増やして群体を形成し、群体同士のつながりはネットワーク上に張り巡らされた血管のつながりである。この群体の構成個数は実験者が物理的に切り離すことによって容易に変更でき、それによってサイズごとの代謝率を調べることができる。さらに、このホヤにはある時間周期ごとに一斉に分裂・増殖を行う"takeover"と呼ばれる現象が見られ、このときすべての個体間の血管つながりが切れ、このときの代謝率は同じ個体数の群体の代謝率よりも大きくなることが示されている。二つ目の例の場合は、これから増殖を行うために代謝率が増加したのだと見ることもできるのであるが、一つ目のサイズと代謝率の間の関係は、そのように簡単に説明されるものではない。これまでのKleiber則の説明としては、体温の維持と表面積との関係から2/3を導くものや、血管構造が大動脈から毛細血管に至るまでがフラクタル的であることに着目し、血液の流速と代謝率の間の関係を仮定して3/4を導くものがあった。しかしはじめの理論では変温動物や単細胞においてもこの関係が成り立つことのうまい説明ができない。また、二つ目の理論についても、より物理的な描像なため説得力はあるが、ホヤの場合、血管構造、特に血管の太さや長さに関して自己相似的とは言えず、また、その理論の中で仮定される"巨視的プールから微視的構造に向かう流れ"というものも見られないため、この理論では説明することはできない。したがって、引用元の記事の中では一つの仮説として以下のようなものが提唱されている。

>『べき乗のサイズ効果は個体同士の局所的な相互作用によって生じ、takeover中には個体間の連絡が切れるため、サイズ効果が見られなくなる。同じユニットが局所的な相互作用をもつ系は、自己組織化臨界状態にある可能性があり、臨界状態とは相互作用に効果がべき関数で記載されるものである。ホヤを含め、動物は自己組織化臨界状態にあるのではないか。』


ここまで述べて来たような、人間の作るシステムの中での最適人数の議論や、生物のアロメトリー則などは、統一した一つの理論として書ける可能性がある。このとき、それまで各モデルの中で重要だと思われてきた要素は、統一された理論の中でのある物理量に対応させることによって、どのモデルでも同じように書けるかもしれない。こうしたことを考えていくうちに、より抽象的にこの問題を考えることには何か意義があるのではという思いを抱くようになった。

この論文では、具体的なイメージとしては例の中で挙げられた会議を題材に議論を進めることにした。会議の進行プロセスを元に確率モデルを作成し、その解析を行って、その結果をまとめた。
