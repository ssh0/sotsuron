\chapter{各モデルにおける結果と考察}

この章では,2章で設定したモデル化の方法を用いて,$x$と$X_{i}$の選択に関するルールと,$x$間のエッジの結び方に関するルールを定め,そのルールに従ったときにはどのようなネットワークの特徴量が得られるかについて,それぞれ見ていくことにする。3.1節では$X_{i}$を等確率で選び,その中から$x$が一様に選ばれる場合,3.2節では$X_{i}$を一つ前の$X_{j}$によって決まる確率で選び,その中から$x$が一様に選ばれる場合を考える。3.3節では,$X_{i}$に依らず過去の点$x$を参照にして次の点を選択する場合を考察し,3.4節では3.3節の場合を拡張し,近距離にある点をクラスター化し,そのクラスタ―$y$同士のネットワークを考えたモデルを考察する。それぞれの場合について,解析的に計算ができるところは計算によって理論値を求め,シミュレーションの結果と対応するかを確かめた。また,それぞれの場合について,特にシステムサイズ$N$とどう関連するのかに重点をおいて考察を行った。

\section{$X_{i}$を等確率で選び,その中から$x$が一様に選ばれる場合}

この設定は,はじめに$N$個の集合$X$から1つの$X_{i}$選択し,点$x$はそれぞれの $X_{i}$について$\Omega = [0,1]$に一様に分布しているとする。
このとき,$X_{i}$の数$N$が変化しても,$x$についての選び方は同じであるから,$x$の作るネットワークだけでは,本来考えたい$N$との間の関係は見ることができない。
しかしながら,選ばれた点のネットワークに関する性質を調べるには良い練習となる。

$X_{i}$から$k$番目に点が選ばれ,その後時刻$k+1$に$X_{j}$から点が選ばれる確率は,$X_{i}$の数$N$として
\[p_{i}(k,j) = p(N) = \frac{1}{N}\]
のように,直前の履歴に依存せず,等確率である場合を考えている。

このとき,$x$は$[0,1]$の間の値を一様な確率で取るとしていたので,そうして得られた確率変数$x$に関して,
確率密度関数$f(x)$は
\[f(x) = 1\ \ \ (0\le x \le 1)\ \ \ \text{otherwise}\ \ 0\]
であり,累積分布関数$F(x)$は
\[F(x) = x\ \ \ (0\le x \le 1)\]
である。

時間発展によって点が選ばれていくごとに,その点と距離$r$より近い位置にある点との間にすべてエッジを張っていくことにする。
すなわち,この問題はある閾値$r(0<r\le1)$を定めた時に,時刻$k$で選ばれた点$x_{k}$を中心とする領域$[x^{k}-r, x^{k}+r]$の中に入るそれまでに出た点の数はいくつか,という問題に帰着される。

すなわち図\ref{fig:f2}のような状況を考えていることになる。
\begin{figure}[H]
    \begin{center}
        \includegraphics[width=10cm]{../simple1/simple001_1.jpg}
        \caption{$[0,1]$の数直線上で閾値$r$で定められる領域}
        \label{fig:f2}
    \end{center}
\end{figure}
一様な確率で$[0,1]$の間の数が選ばれるとき,その確率変数が$[\max(0,x-r), \min(x+r,1)]$の範囲に入っている確率は,確率密度関数を用いて,
\begin{align}
p(\max(0, x-r), \min(x+r, 1)) &= \int ^{\min(x+r,1)}_{\max(0, x-r)} 1 dx \nonumber \\
&= \left[ x\right]^{\min(x+r,1)}_{\max(0, x-r)}
\end{align}
よって
\begin{eqnarray}
p(x,r)= \left\{ \begin{array}{ll}x+r & 0\le x< \min(r,1-r) \nonumber \\
p(r) = \min(2r, 1) & \min(r, 1-r)\le x \le \max(1-r, r) \nonumber \\
1 - x+r & \max(1-r, r) < x \le 1
\end{array}\right.
\end{eqnarray}
を得る。

これはグラフにすると,図\ref{fig:f3}のようになる。
\begin{figure}[H]
    \begin{center}
        \includegraphics[width=10cm]{../img/fig3.png}
        \caption{$r=0.25$のとき,それぞれの位置$x$において他の点を1つ見出す確率$p(x)$}
        \label{fig:f3}
    \end{center}
\end{figure}

$k+1$番目に点$x_{k+1}$が選ばれたとき,$k$番目までに選ばれた点のうち$y$個の点が領域$[\max(0,x-r), \min(x+r,1)]$の中に存在する確率は,
\begin{eqnarray}
_{k}C_{y}p(r)^{y}p(r)^{k-y}
\label{eq:e1}
\end{eqnarray}
で表せる。
ここで注意すべき点は,$p(x_{k}, r)$は$x_{k}$によって変わるものであったから,$x$に対して期待値を取ったものを考えなければならないことである。
$p(r)$はそのようにして得られた期待値であることを意味している。

$p(r)$を求めると,
$0<r\le0.5$のとき
\begin{align}p(r) = E(p(x, r)) &= \int^{r}_{0}x+r\mathrm{d}x + \int^{1-r}_{r}2r \mathrm{d}x + \int^{1}_{1-r} 1-x+r \mathrm{d}x\nonumber \\
&= \left[\frac{x^{2}}{2} + rx \right]^{r}_{0} + \left[ 2rx\right]^{1-r}_{r}+ \left[ x-\frac{x^{2}}{2} + rx\right]^{1}_{1-r}\nonumber \\
&= -r^{2} + 2r \end{align}
$0.5<r\le1$のときも同様にして
\[p(r) = E(p(x,r)) = -r^{2} + 2r\]
である。

また,式(\ref{eq:e1})は
\[P[X=y] =\ _{k}C_{y}p(r)^{y}(1-p(r))^{k-y}\]
のように書けば明らかなように,確率変数$X$に対するパラメータ$k,p$の二項分布$B(k,p)$を表している。

この時,確率変数$X$に対する期待値と分散は,
\[E(X) = kp(r)\]
\[V(X) = kp(r)(1-p(r))\]
である。

時刻$k$が大きい時,具体的には
\[kp(r) > 5,\]
\[kp(r)(1-p(r)) > 5\]
をみたす$k$のとき,二項分布は正規分布に近似できるので,期待値は中央値とほぼ等しくなる。
すなわち,$y=kp(r)$より多くの点を見出す確率$P[X\le y]$は,どんな時刻$k$においても$1/2$となる。

すなわち,時刻の依存性はないため,$x_{k+1}$のまわりの領域に$y$個以上の他の点が存在する場合を1,存在しない場合を0とすれば,これらの起こる確率は,それぞれ$1/2$であり,$k$回目に$y$個以上のエッジが張られる確率は,$p=1/2$の幾何分布に従う:
\[P[X=k] = \left( \frac{1}{2} \right)^{k}\]
このときの期待値と分散は
\[E(X) = \frac{1}{p} = 2\]
\[V(X) = \frac{1-p}{p^{2}} = 2\]

すなわち,これまでのような場合を考えたとすると,平均として2回で$y$個以上のエッジが張られ,ほとんどの試行は$1\sim3$回でその値に到達することが分かる。
しかし,これは先程述べたような二項分布の正規分布への近似ができない領域であることに注意が必要である。

\section{$X_{i}$を一つ前の$X_{j}$によって決まる確率で選び,その中から点$x$が一様に選ばれる場合}

次に考えるのは,$X_{i}$の選び方が,一つ前の$X$による確率で決定するような場合である。
この確率を定めるにあたり,会議として自然と思われる$X_{i}$間に距離が定義でき,その距離にしたがって確率が決まるような問題を考えることとした。
このとき距離の計算に用いることができるパラメータの数を$b$とし,距離として$b$次元ユークリッド距離を考えることにする。

すなわち$b$個のパラメータを要素とする元からなる空間$X$があったとき,
距離関数$d: X \times X \rightarrow \mathbb{R}$が
\[d(x, y) = \sqrt{\sum_{i=1}^{b}(x_{i}-y_{i})^{2}}\ \ ,\ x,y\in X\]
と書けることを意味する。
実際の場合には,各データ同士の相関を考慮に入れたマハラノビス距離などのほうが適当な場合もあるかもしれないが,まずはイメージしやすいということでユークリッド距離を考えた。
以下では,記述を簡単にするため,$X_{i}$と$X_{j}$の間の距離を$d_{ij}$と書くことにする。

時刻$k$に$X_{i}$が選ばれ,その後時刻$k+1$に$X_{j}$から点$x_{k+1}^{j}$が選ばれる確率$p_{k}(i,j)$は,距離$d_{ij}$の関数として,次のようにできる。
\[p_{k}(i,j) = \frac{g_{k}(d_{ij})}{\sum_{j} g_{k}(d_{ij})}\]
ここでの関数$g$の選び方によって,距離の大きさがどのように確率に重みを持たせるかということが決定される。
一般に$g$は時刻$k$によって変化してもいいので,添字$k$をつけて時刻$k$における関数であることを表した。

単純な例として$g_{k}(d) = const.,\ ^{\forall}k, d\in \mathbb{R}^{1}$とすると,距離に依らず$X_{i}$が選ばれるわけなので,3.1節の$X_{i}$の選び方と同じである。
$g(d)$は$[0, +\infty]$で定義される非負の実関数であればよい。

ex)
\[g(d) = \frac{1}{d+1}\]
\[g(d) = e^{-d}\]
\[g(d) = \left\{ \begin{array}{ll} c & (0\le d \le 1/c) \nonumber\\
1/d & (d>1/c)\end{array}\right., \ \ c>0\]

しかし,ここで注意すべき点として,どの$X_{i}$が選ばれたとしても,3.1で考えたように,どの$X_{i}$も$[0,1]$から一様に点$x$を取るから,結局点$x$について見たときの試行は同様のことをしており,どの$X_{i}$が選ばれるかは本質的な問題にはならないことが分かる。

これまでの設定を用いて数値シミュレーションを行った結果を図\ref{fig:f4}$\sim$\ref{fig:f6}に示す。
シミュレーションでは,確率を決める距離の関数$g(d)$として
\[g(d) = e^{-d}\]
を採用し,$x$の次元$a=2$,$X_{i}$の数$N=6$と設定した。

図\ref{fig:f4}の中の青い丸が$X_{i}$の位置を表しており,それぞれの丸の大きさは,一回の試行において選択された頻度を表したものとなっている。
円の間に張られた線分とそこに記された数字の組は,1番目の数字をラベルとしてもつ$X_{i}$のあとに2番目の数字をラベルとしてもつ$X_{j}$が選ばれたことを意味している。
図\ref{fig:f5}で青色の曲線で表されているのは,時刻$k$に選ばれた点$x$のまわりの$r$で決まる領域の中に入った,それまでに選択された点の数である。
また,このグラフで緑色の直線として表されているものは,3.1のように計算で求めた値であり,
\[l = (-r^{2} + 2r)k\]
であった。
このグラフを見て分かるように,理論値と実験値はよく一致していることが分かる。
同じようにして偏差についても計算ができており,
$V(l) = \sqrt{(-r^{2} + 2r)(1+r^{2}-2r)k}$
である。これはグラフにおいて緑色の領域として描かれている。
図\ref{fig:f6}は,時刻$k$までに張られたエッジの数の総和を表しており,この中の緑色の曲線も,計算で求めることのできるものであった:
\[L = \frac{1}{2}(-r^{2} + 2r)k^{2}\]
この値もグラフから分かるように,よく理論値と一致していることが分かる。

\begin{figure}[H]
    \begin{center}
        \includegraphics[width=10cm]{../download2_2.png}
        \caption{$X_{i}$の選択された頻度と$X_{i}$間ネットワーク}
        \label{fig:f4}
    \end{center}
\end{figure}
\begin{figure}[H]
    \begin{center}
        \includegraphics[width=10cm]{../download2_3.png}
        \caption{時刻$k$とその時張られたエッジの本数$l$との間の関係}
        \label{fig:f5}
    \end{center}
\end{figure}
\begin{figure}[H]
    \begin{center}
        \includegraphics[width=10cm]{../download2_4.png}
        \caption{時刻$k$までに張られたエッジの数の総和$L$}
        \label{fig:f6}
    \end{center}
\end{figure}

また,図\ref{fig:f5}における直線の傾きは$r$に依存していたがこの傾きを$r$に関してプロットすると,以下の図\ref{fig:f7}のようになる。

\begin{figure}[H]
    \begin{center}
        \includegraphics[width=10cm]{../download2_5.png}
        \caption{図\ref{fig:f5}の理論式直線の傾きと$r$の関係}
        \label{fig:f7}
    \end{center}
\end{figure}

図\ref{fig:f4}で,$r=1/3$としていたので,緑の直線の傾きは$-1/3(1/3-2) = 5/9$である。

\subsection{過去の点$x$を参照にして次の点を選択する場合}

1の場合には、$X_{i}$を選ぶ確率とそのうえで$x$が選ばれる確率は独立であるというものであった。しかし、これまで考えたように、単にそのようなモデルを考えただけだと$X_{i}$の数$N$の効果をうまく反映できないように思える。したがって、次に考えるモデルは前に選ばれた点に近い点が選ばれることにし、そのときその点をもつ$X_{i}$が選ばれたとするモデルである。

$X_{i}$はそれぞれ$s_{i}$個の点をもっており、モデルの設定時に仮定したとおり、これらの点は異なる$X$同士で共有されることはない。はじめに点$x_{0}$が与えられ、次に時刻$1$では、それぞれの$X$の中で最もその点に近いものを選び、より近いものをもった$X$の順に整列する。この$X$の順番にしたがって、$X_{i}$にそれぞれに割り当てられた確率$P_{i}$で、実際にその点$x$が選択されるかどうかが決定する。点が選択されない場合(確率$1-P_{i}$)ときは、$X$の順番で次の順番になっているものについて、同様の試行を繰り返す。もしすべての$X$について点$x$が選択されないときは、時刻$k$を一つすすめ、その時刻には$x_{0}$が選ばれたとする。
\begin{figure}[H]
    \begin{center}
        \includegraphics[width=12.5cm]{../img/figure2.jpg}
        \caption{$[0,1]$の数直線上で閾値$r$で定められる領域}
        \label{fig:f8}
    \end{center}
\end{figure}

図\ref{fig:f8}を用いて説明する。中心にある"pre"と名のついた点が参照する一つの点であり、この次の時刻に選ばれる点の選び方は、まずそれぞれの$X_{i}$について"pre"に最も近い点を選び、その近さの順に順番を付けることにする。図で青枠内の緑の矢印に示された数字がそれぞれの$X_{i}$の順番である。この順番にしたがってそれぞれの$X_{i}$に割り当てられた$P_{i}$にしたがって実際にその点が選ばれるかどうかが決まる。右下の矢印はそのことを表したフローチャートになっている。すべての$X_{i}$について点が選ばれなかった場合、時刻を一つ進め、はじめの点$x_{0}$を選択する。

参照する点の選び方として、以下のような場合分けを考えた。

\begin{enumerate}
    \item なし (case 1)
    \item 一つの点
    \begin{enumerate}
        \item 時刻0における点 (case 2)
        \item 一つ前の時刻の点 (case 3)
        \end{enumerate}
    \item 二つの点
    \begin{enumerate}
        \item 時刻0における点 + 一つ前の時刻の点 (case 4)
        \item 二つ前の時刻までの点 (case 5)
        \end{enumerate}
\end{enumerate}

\subsubsection{過去の点の影響を受けない場合 (case 1)}

$X_{i}$が$X$の配列の中で$r+1$($r = 1, 2, \cdots , n-1$)番目に選ばれたとき、$X_{i}$まで順番が回ってくる確率は
$$p_{r+1}(i) = \frac{\sum_{J = \langle j_{0}, \cdots ,j_{r-1} \rangle _{r}}\prod_{j\in J}(1-P_{j})}{_{n-1}C_{r}}.$$
ここで$J = \langle j_{0}, j_{1}, \cdots ,j_{r-1} \rangle_{r}$は、$i$を除く$n-1$個の要素から$r$個選んだときの組み合わせのうちの1揃いをあらわすことにする。

また、1番目に$x$を選択する権利を得たときに、選択権が回ってくる確率は当然
$$p_{1}(i) = 1$$
である。

説明のための具体的な例として、$N = 5, n = 5, i = 1, r = 2$とすると、$X_{1}$までに2つの$X$が選択権を得ているはずであり、その2つの組み合わせは$(0,2), (0,3), (0,4), (2,3), (2,4), (3,4)$の6つの組み合わせがある。上の式では$J$の一つは$(0, 2)$であり、このとき$j_{0} = 0, j_{1} = 2$である。この$J$に関して和をとり、組み合わせの数$\ _{4}C_{2} = 6$で割って期待値を求めている。
\begin{align}
p_{r+1}(1) &= \left[(1-P_{0})(1-P_{2}) + (1-P_{0})(1-P_{3}) + (1-P_{0})(1-P_{4}) \right. \nonumber \\
&\ \left. + (1-P_{2})(1-P_{3}) + (1-P_{2})(1-P_{4}) + (1-P_{3})(1-P_{4}) \right]/6
\end{align}
$X_{i}$が選択権の順番で$r$番目になる確率は等しいので、$r$に関する平均をとり、$P_{i}$をかければ、これは$X_{i}$から点が選ばれる確率の期待値となる。

$$p(i) = \frac{\sum_{r=0}^{n}p_{r}(i)P_{i}}{n}.$$

このとき得られた確率は$X_{i}$によって異なり、期待値としては毎時刻ごとにそれぞれの$X_{i}$がその確率が点が選択されることになり、単純な確率過程に帰着できる。


\subsubsection{1つ前の点を参照する場合}

それぞれの$X_{i}$が、配列の順番が回ってきたときに同じ確率$p$で点を選ぶとしたとき、はじめに与えられた点$x_{0}$のみを参照にしてその点からの近さのみで次の点を選ぶ場合と、一つ前の点のみを参照にして次の点を選ぶ場合の二つの場合に関してシミュレーションを行った。このとき、シミュレーション時に変更できるパラメータとしては、$X_{i}$の数$N$、$X{i}$あたりにもつ点の数$S$、順番が回ってきたときに点を選択する確率$p$がある。

点$x$と$y$の間の近さの指標として、$a$次元ユークリッド距離

\begin{align}D(x, y) &= d(x,y)\\
&= \sqrt{(x_{1} - y_{1})^{2} + (x_{2} - y_{2})^{2} + \cdots + (x_{a} - y_{a})^{2}}\end{align}

を用いることにする。シミュレーションでは、描画の簡単さとイメージのしやすさから$a=2$の場合を考えることにした。

以下に作成したプログラムを用いて得られたネットワークの例を示す(図\ref{fig:f9}, 図\ref{fig:f10})。
図\ref{fig:f9}は、はじめに与えられた点$x_{0}$のみを参照にしたときに得られたネットワークのグラフであり、このとき、中心付近の青色の点は$x_{0}$であり、この点とつながっているエッジは薄い灰色で描画されている。それ以外の黒い線分は、選ばれた意見同士のエッジを表しており、ノードに振られた数字はその点が選ばれた時刻$k$を示している。

\begin{figure}[H]
    \begin{center}
        \includegraphics[width=12.5cm]{../simple3/case_2.jpg}
        \caption{$x_{0}$のみを参照にして次の点を選んだ場合のシミュレーション結果の一例}
        \label{fig:f9}
    \end{center}
\end{figure}

\begin{figure}[H]
    \begin{center}
        \includegraphics[width=12.5cm]{../simple3/case_3.jpg}
        \caption{1つの点のみを参照にして次の点を選んだ場合のシミュレーション結果の一例}
        \label{fig:f10}
    \end{center}
\end{figure}

\subsubsection{二つの点を参照して次の点を選択する場合}

次に、過去の二つの選択された点を参照して、その2点から近い位置にあるてんを次に選ばれる点の候補とする場合を考えることにする。このとき、2点からの近さの指標は、うまくこちらで決めてやる必要がある。近さの指標となる一つの例として、二つの点を焦点とした楕円を考えることができる。すなわち2つの点からの距離の和が等しい点はすべて同じ距離にあると見なすような近さが考えられる。したがって2点$(y, z)$からの点$x$の近さの指標$D(x, (y, z))$は、通常の$a$次元ユークリッド次元を$d(x,y)$と表すと、
$$D(x, (y, z)) = d(x,y) + d(x, z)$$
のように書けることになる。また、この考えが役に立つのは、それぞれの距離を足すときに適当な正の係数を掛けることによって、2つの点のどちらへの近さを優先するのかという条件を付け加えることができる点である。すなわち、正の定数$\alpha$ ,$\beta$を用いて、
$$D(x, (y, z)) = \alpha d(x,y) + \beta d(x, z)$$
のようにも書くことができる。参照する点をさらに増やしたい場合にも、それらの点との間の距離に係数を掛けて足せばよいので、いくらでも参照点を増やすことは可能である:
$$D(x_{k+1}, (x_{1}, x_{2}, \cdots , x_{k})) = \sum_{i=1}^{k}w_{i}d(x_{k+1}, x_{i})\ \ (w_{i} > 0)$$
また、別の近さの指標として、ベクトル$\vec{Y} = y-x$, $\vec{Z} = z-x$として
$$D(x, (y,z)) = |t\vec{Y} + (1-t)\vec{Z}|\ \ (0 \le t \le 1)$$
とする方法もある。このとき、右辺の括弧の内部が表すベクトルは、点$y$,$z$を結んだ線分$YZ$を$(1-t):1$に内分する点のベクトルを示しており、$D$はその点から$x$までのユークリッド距離を表すことになる。

シミュレーションでは、はじめに挙げた近さの指標を用いて選択権の順番を決定することにする、図\ref{fig:f11}と図\ref{fig:f12}は、それぞれ$x_{0}$の点と一つ前の点を参照にして点を選択する場合と、二つ前までの点を参照にして点を選択する場合のシミュレーションを行ったときに得られた結果の例である。

\begin{figure}[H]
    \begin{center}
        \includegraphics[width=10cm]{../simple3/case_4.jpg}
        \caption{$x_{0}$と一つ前の点を参照にして次の点を選んだ場合のシミュレーション結果の一例}
        \label{fig:f11}
    \end{center}
\end{figure}
\begin{figure}[H]
    \begin{center}
        \includegraphics[width=10cm]{../simple3/case_5.jpg}
        \caption{二つ前までの点を参照にして次の点を選んだ場合のシミュレーション結果の一例}
        \label{fig:f12}
    \end{center}
\end{figure}

\section{近距離の点をクラスター化するモデル}

このモデルでは,3.2で考えたモデルと同様に,点$x$をそれまでに選択された点からの近さで選んでゆくようなモデルを考えることにする。

これまで考えてきたモデルとの大きな相違は,点を選択していく過程をは始める前に,すべての$x$について,自分から近い位置にある点同士をエッジで結んでいき,クラスターを形成することである。こうして得られたクラスター$y$について,その一つ前に選ばれた点$x$から最も近い点を含む$y$から,次の時刻の点が選ばれるとする。このとき$y$の中から点を選ぶ方法は,そのクラスター内に含まれる点の属するそれぞれの$X$に割り当てられた重みによって決定されるとする。

このモデルの着想を得るにあたって,実際の話し合いの際に,人が多くなることによって人がどのように考えるか,ということに影響を受けている。すなわち,話し合いに参加する人が多いほど,自分の意見と同じ様な意見をもつ人がいるだろう,と思う効果であり,$X_{i}$に割り当てられた重みは,個人の発言力をあらわしていると考えることができる。

シミュレーションにおいてクラスター化を行う際には,自分と他の点との間の距離を測り,これが$r$より小さいものの間にエッジを張ることにする。このとき,(特に$r$が小さいときに)効率よく近傍点を探すことができるよう,はじめに全体を長さ$l(>r)$の正方形の領域(セル)に分割し,それぞれの点がどのセルに属しているかを記録しておく。このようにすると,近傍点を探す際には,自分の属するセルとその周囲8マスを含めた9つのセルの中にある点のみについて調べればよい。このようにしてすべての点について順番に閾値$r$の内部にある点を選択していく。このとき領域内に入ったすべての点にクラスター番号が与えられていない時には,通し番号でクラスター番号を割り当てる。閾値$r$で定められる領域内に,より小さいクラスター番号をもつ点が存在する時には,その中で一番小さいクラスター番号を,接続しているすべての点に同じクラスター番号を付与する。このようにすれば,既に存在するクラスターとの間の融合などを考慮に入れたクラスター化が行える。また,今回のシミュレーションでは,$X_{i}$の違いによる選択確率の違いはないものとした。

図\ref{fig:f15}に,$r=0.07, K=20, N=4, S=20$としたときに実際にどのようにクラスターが形成され,点が選択されていくかの様子を示した図を載せる。グラフで薄いグレーで描かれた線はノード間の距離が$r$より小さいために張られたエッジであり,これが連結しているひとまとまりがクラスターである。青色の線は,選ばれた点同士をつなぐ(有向)エッジであり,振られている番号は時刻$k$に張られたエッジであることを表す。
\begin{figure}[H]
    \begin{center}
        \includegraphics[width=10cm]{../img/cluster.png}
        \caption{近傍点をクラスター化するモデルのシミュレーション結果の例}
        \label{fig:f15}
    \end{center}
\end{figure}

このシミュレーションでは,考えるべき問題が二つある。一つは$r$によって作られるクラスターに関する性質である。もうひとつは,選ばれた点によって作られた軌跡に関するものだである。ただし,それぞれが独立でないために,完全に分けて考えることはできない。

まずは,閾値$r$と,点の密度に関係してクラスターがどのように形成されるかについての議論をしていくことにする。

図\ref{fig:f16}は閾値$r$を変えたときの,点の総数に対するクラスターの数の関係を示している。横軸$r$,縦軸$\phi = 1- (\text{クラスターの数})/(\text{点の総数})$として,通常の軸でのプロット(上段)と両対数プロット(下段)を取っており,縦軸の値は100回の試行の平均をとったものとなっている。
\begin{figure}[H]
    \begin{center}
        \includegraphics[width=10cm]{../img/r_phi_1.png}
        \caption{横軸$r$,縦軸$\phi = 1- (\text{クラスターの数})/(\text{点の総数})$としたグラフ}
        \label{fig:f16}
    \end{center}
\end{figure}
両対数プロットを見て分かるように,$r$の小さい領域では,ベキで近似することができそうであることが分かる。図\ref{fig:f17}には,この両対数グラフの$0<r<0.07$の範囲を直線でフィッティングしたものを示す。このときの傾きは,およそ1.86であった。
\begin{figure}[H]
    \begin{center}
        \includegraphics[width=10cm]{../img/r_phi_1_power.png}
        \caption{図\ref{fig:f16}の$0<r<0.07$の範囲をべき乗に近似したもの}
        \label{fig:f17}
    \end{center}
\end{figure}

また,これはS字型の曲線(シグモイド曲線)なので,その代表的な関数系である
\begin{eqnarray}\phi (r) = 1 - \exp \left[ -  \left( \frac{r}{\omega} \right)^{2} \right]\label{eq:e4}
\end{eqnarray}
としてパラメータ$\omega$に関して最小2乗法でフィッティングを行った。このときは先ほどの場合とは異なり,$r$の比較的大きい領域のデータを含んでいてもよい。得られたパラメータの値は$\omega=0.0715$ほどであり,図\ref{fig:f17}でべきで近似した場合に比べて,よくフィッティングできている。
\begin{figure}[H]
    \begin{center}
        \includegraphics[width=10cm]{../img/r_phi_1_sigmoid.png}
        \caption{図\ref{fig:f16}をシグモイド曲線(\ref{eq:e4})に近似したもの}
        \label{fig:f18}
    \end{center}
\end{figure}
また,(\ref{eq:e4})式は$\phi = 1- (\text{クラスターの数})/(\text{点の総数})$の形との整合もとれているように思われる。

% =============================================================================

次に、$X_{i}$の数が変化したときにステップ間の平均移動距離$\phi$がどう変化するかについて見てみることにする。このとき、点$x$の総数$M$と$X_{i}$の数$N$、$X_{i}$あたりの点の数$S$の間には比例の関係($M=N\times S$)が成り立っており、$N$を増やすことと$S$を増やすことはこの場合等価であるから、より細かく値を刻むことのできる$S$を変化させたときのクラスター数との間の関係について調べた。横軸を$S$、縦軸を平均のステップ間距離$\phi$としたグラフを図\ref{fig:f19}に示す。このときの$N$は$N=6$であり、$r=0.07$で、100回の試行を平均したものとなっている。
\begin{figure}[H]
    \begin{center}
        \includegraphics[width=12.5cm]{../img/S_phi_1.png}
        \caption{$S$を変化させたときのステップ間の平均距離}
        \label{fig:f19}
    \end{center}
\end{figure}

\subsection{解析的な計算}
以下に示すのは、これまで調べてきた性質が、解析的な計算によって求められないかと考えて行った試行である。

まず、点の分布する範囲は$\Omega = [0,1]\times [0,1]$であり、この中の面積$S$の領域の中に点を見出す確率は$S$である。

次に、領域内のある点$\vec x=(x,y)$を中心として半径$r$($0 < r \le 0.5$)の領域$B(\vec x, r)$内に点を見出す確率$p(\vec x)$は、境界の影響をうけない領域($\Omega ' = \{(x,y) | r \le x \le 1-r, r \le y \le 1-r \}$)では$\pi r^{2}$であり、境界の影響を受ける領域($\Omega'' = \Omega / \Omega'$)において点を見出す確率は、領域を
\begin{align}
\Omega''_{x} &= \{(x,y) | 0 \le x < r, r< y < 1-r\}\nonumber \\
\Omega''_{1-x} &= \{(x,y) | 1-r < x \le 1, r< y < 1-r\}\nonumber \\
\Omega''_{y} &= \{(x,y) | r < x < 1-r, 0 \le y < r\}\nonumber \\
\Omega''_{1-y} &= \{(x,y) | r < x < 1-r, 1-r < y \le 1\}\nonumber \\
\Omega''_{x,y} &= \{(x,y) | 0 \le x < r, 0 \le y < r\}\nonumber \\
\Omega''_{1-x,y} &= \{(x,y) | 1-r < x \le 1, 0 \le y < r\}\nonumber \\
\Omega''_{x,1-y} &= \{(x,y) | 0 \le x < r, 1-r < y \le 1\}\nonumber \\
\Omega''_{1-x, 1-y} &= \{(x,y) | 1-r < x \le 1, 1-r < y \le 1\}\nonumber
\end{align}
のようにあらわして、それぞれ考えることにする(図\ref{fig:f20})。
\begin{figure}[H]
    \begin{center}
        \includegraphics[width=12.5cm]{../img/omega.jpg}
        \caption{領域$\Omega$を分割した各領域}
        \label{fig:f20}
    \end{center}
\end{figure}
$\Omega''_{i} = \{\Omega''_{x}, \Omega''_{1-x}, \Omega''_{y}, \Omega''_{1-y}$\}、また$\Omega''_{i,j} = \{\Omega''_{x,y}, \Omega''_{1-x,y}, \Omega''_{x,1-y}, \Omega''_{1-x,1-y}\}$でまとめて書くことにすると、
\[p(r)_{\Omega''_{i}} = i \sqrt{r^{2}-i^{2}} + r^{2} \left[ \pi -\arccos \frac{i}{r} \right]\]
\begin{figure}[H]
    \begin{center}
        \includegraphics[width=12.5cm]{../img/omega_x.jpg}
        \caption{$\vec{x} \in \Omega''_{x}$であるときの面積$S$の求め方}
        \label{fig:f21}
    \end{center}
\end{figure}
\begin{align}p(r)_{\Omega''_{i,j}} = &\frac{1}{2}\left\{ \sqrt{r^{2}-i^{2}} + \min \left(j, \sqrt{r^{2}-i^{2}}\right) \right\}i + \frac{1}{2}\left\{ \sqrt{r^{2}-j^{2}} + \min \left( i, \sqrt{r^{2}-j^{2}}\right) \right\}j \nonumber \\
&+ \frac{1}{2}r^{2} \left\{ 2\pi -\arccos \frac{i}{r}-\arccos \frac{j}{r}-\min \left( \frac{\pi}{2}, \arccos \frac{i}{r} +\arccos \frac{j}{r} \right) \right\}
\end{align}
のようにあらわすことができる。
\begin{figure}[H]
    \begin{center}
        \includegraphics[width=12.5cm]{../img/omega_xy.jpg}
        \caption{$\vec{x} \in \Omega''_{x,y}$のときの面積$S$の求め方の一例}
        \label{fig:f22}
    \end{center}
\end{figure}

確率$p$をすべての領域について積分した値は、領域$\Omega$から一様乱数によって一つの点を選び、その点を中心とした$r$による範囲に1つの点を見出す確率の期待値となる。
この確率を$p'(r)$とし、$0\le r \le 0.5$のときは
\[p'(r) = p'(r)_{\Omega''} + 4p'(r)_{\Omega''_{i}} + 4p'(r)_{\Omega''_{i,j}}\]
とできる。それぞれの領域について積分を実行する。
\[p'(r)_{\Omega'} = \int_{r}^{1-r} \int_{r}^{1-r}\pi r^{2}\mathrm{d}x\mathrm{d}y = (1-2r)^{2}\pi r^{2}\]
\begin{align}
p'(r)_{\Omega'_{i}} = p'(r)_{\Omega'_{x}} &= \int_{0}^{r} \int_{r}^{1-r}\mathrm{d}x\mathrm{d}y\ x\sqrt{r^{2}-x^{2}} + r^{2}\left[\pi - \arccos\frac{x}{r}\right]\nonumber \\
&= (1-2r)\left\{ \frac{r^{3}}{3} + r^{2}\pi\cdot r - r^{2}\cdot r \right\}\nonumber \\
&= (1-2r)r^{3}\left( \pi-\frac{2}{3} \right)
\end{align}

NOTE1:
\begin{align}
&\int_{0}^{r}\mathrm{d}x\ x\sqrt{r^{2}-x^{2}} \nonumber \\
&\ \ \ \ \ \left[x = r\cos \theta \right]\nonumber \\
&= \int_{\frac{\pi}{2}}^{0}\mathrm{d}\theta\ (-r\sin\theta)\ r\cos\theta\ r\sin\theta\nonumber \\
&= r^{3}\left[ \frac{\sin^{3}\theta}{3}\right]^{\frac{\pi}{2}}_{0} \nonumber \\
&= \frac{r^{3}}{3}
\end{align}

NOTE2:
$x = \cos t \ (0< t< \pi)$とすると
\[\frac{\mathrm{d}x}{\mathrm{d}t} = - \sin t < 0\]
$t = \arccos x$であるから、
\begin{align}
\frac{\mathrm{d}}{\mathrm{d}x}\arccos x &= \frac{1}{\frac{\mathrm{d}}{\mathrm{d}t}\cos t} = -\frac{1}{\sin t}\nonumber \\
&=- \frac{1}{\sqrt{\sin^{2}t}} = - \frac{1}{\sqrt{1- \cos^{2}t}} \nonumber \\
&= - \frac{1}{\sqrt{1- x^{2}}}
\end{align}
したがって、
\begin{align}
\int \arccos x \mathrm{d}x\  &= x\arccos x + \int \frac{x}{\sqrt{1-x^{2}}}\mathrm{d}x\nonumber \\
&=x\arccos x - \sqrt{1-x^{2}} + C
\end{align}
($C$は積分定数)

今の場合、
\begin{align}
\int^{r}_{0} \arccos \frac{x}{r} \mathrm{d}x &= \int^{1}_{0}\arccos t \cdot r\mathrm{d}t\nonumber \\
&= r \left[ t \arccos t - \sqrt{1-t^{2}} \right]^{1}_{0}\nonumber \\
&= r ( 1\arccos1 -\sqrt{1-1} - 0 \arccos0 + \sqrt{1-0})\nonumber \\
&= r
\end{align}

$p'(r)_{\Omega''_{i,j}}$を以下のように分解してそれぞれ計算する。
\begin{align}
p'(r)_{\Omega''_{i,j}} &= p'(r)_{\Omega''_{x, y}}\nonumber \\
&= p'(r)_{\Omega''_{x, y}1}  +p'(r)_{\Omega''_{x, y}2}\nonumber \\
&= p'(r)_{\Omega''_{x, y}1'} - p'(r)_{\Omega''_{x, y}1''} + p'(r)_{\Omega''_{x, y}2}
\end{align}
\begin{align}
p'(r)_{\Omega''_{x,y}1'} &= \int^{r}_{0}\int^{r}_{0}\mathrm{d}x\mathrm{d}y\ x\sqrt{r^{2}-x^{2}} + y \sqrt{r^{2} -y^{2}} + \frac{1}{2}r^{2}\left( 2\pi -2\arccos\frac{x}{r} -2\arccos\frac{y}{r} \right) \nonumber \\
&= r\int^{r}_{0}\mathrm{d}x\ \left\{x\sqrt{r^{2}-x^{2}} - r^{2}\arccos\frac{x}{r} \right\} + r\int^{r}_{0}\mathrm{d}x\ \left\{x\sqrt{r^{2}-x^{2}} - r^{2}\arccos\frac{x}{r} \right\} + \pi r^{2}\cdot r^{2}\nonumber \\
&= r\left( \frac{r^{3}}{3} -r^{3} \right) + r\left( \frac{r^{3}}{3} -r^{3} \right) + \pi r^{4}\nonumber \\
&=  \left(\pi -\frac{4}{3}\right)r^{4}
\end{align}
\begin{align}
p'(r)_{\Omega''_{x,y}1''} &= \int^{r}_{0}\int^{\sqrt{r^{2}-x^{2}}}_{0}\mathrm{d}x\mathrm{d}y\ \left[ x\sqrt{r^{2}-x^{2}} + y \sqrt{r^{2} -y^{2}} + r^{2}\left(\pi -\arccos\frac{x}{r} -\arccos\frac{y}{r}\right) \right]\nonumber \\
&= \int^{r}_{0}\mathrm{d}x\left[ \left\{ x\sqrt{r^{2}-x^{2}} + r^{2}\pi -r^{2}\arccos\frac{x}{r} \right\}\sqrt{r^{2}-x^{2}} \right. \nonumber\\
&\ \ \ \ \ \ \ \ \ \ \ \ \ \ + \left. \int^{\sqrt{r^{2}-x^{2}}}_{0}\mathrm{d}y\ y\sqrt{r^{2}-y^{2}} -r^{2}\arccos\frac{y}{r}\right]\nonumber \\
&= \int^{r}_{0}\mathrm{d}x\left[ x(r^{2}-x^{2}) + r^{2}\pi\sqrt{r^{2}-x^{2}} -r^{2}\arccos \frac{x}{r} \sqrt{r^{2}-x^{2}} \right. \nonumber \\
&\ \ \ \ \ \ \ \ \ \ \ \ \ \ + \left. \frac{1}{3}r^{3} - \frac{1}{3}x^{3} - r^{2}\frac{\pi}{2}\sqrt{r^{2}-x^{2}} + r^{2}\arccos\frac{x}{r}\sqrt{r^{2}-x^{2}} +r^{2}x - r^{3}\right] \nonumber \\
&= \int^{r}_{0}\mathrm{d}x \left[-\frac{4}{3}x^{3} + 2r^{2}x - \frac{2}{3}r^{3} + \frac{\pi}{2}r^{2}\sqrt{r^{2}-x^{2}} \right]\nonumber \\
&= \left[ -\frac{4}{3}\frac{x^{4}}{4} + r^{2}x^{2} - \frac{2}{3}r^{3}x \right]^{r}_{0} + \frac{\pi}{2}r^{2}\cdot \frac{1}{4}\pi r^{2}\nonumber \\
&= -\frac{r^{4}}{3} + r^{4} -\frac{2}{3}r^{4} + \frac{\pi ^{2}}{8}r^{4}\nonumber \\
&= \frac{\pi^{2}}{8}r^{4}
\end{align}

NOTE1:
$y=r\cos \theta$とおく。
$y$の積分領域$[0, \sqrt{r^{2}-x^{2}}]$は$\theta$の範囲としては$[\pi/2, \theta' = \arcsin(x/r)]$となる。

\begin{align}
\int^{\sqrt{r^{2}-x^{2}}}_{0}\mathrm{d}y\ y\sqrt{r^{2}-y^{2}} &= \int^{\theta'}_{\frac{\pi}{2}}-r^{3}\sin^{2}\theta\ \cos\theta \mathrm{d}\theta\nonumber \\
&= r^{3}\left[ \frac{\sin^{3}\theta}{3} \right]^{\frac{\pi}{2}}_{\theta'}\nonumber \\
&= \frac{r^{3}}{3} - r^{3}\frac{\left( \frac{x}{r} \right)^{3}}{3}\nonumber \\
&= \frac{r^{3}}{3} - \frac{x^{3}}{3}
\end{align}

NOTE2:
$t = x/r$とおいて積分する。
\begin{align}
r^{2}\int^{\sqrt{r^{2}-x^{2}}}_{0}\mathrm{d}y\ \arccos \frac{y}{r} &= r^{2}\int^{\frac{\sqrt{r^{2}-x^{2}}}{r}}_{0}r\mathrm{d}t\ \arccos t\nonumber \\
&= r^{3}\left[ t\arccos t - \sqrt{1-t^{2}} \right]^{\frac{\sqrt{r^{2}-x^{2}}}{r}}_{0}\nonumber \\
&= r^{3}\left[ \frac{\sqrt{r^{2}-x^{2}}}{r}\left( \frac{\pi}{2} - \arcsin \frac{x}{r} \right) - \frac{x}{r} + 1 \right]\nonumber \\
&= r^{2}\frac{\pi}{2}\sqrt{r^{2}-x^{2}} - r^{2}\sqrt{r^{2}-x^{2}}\arccos\frac{x}{r} - r^{2}x + r^{3}
\end{align}
\begin{align}
p'(r)_{\Omega''_{x, y}2} &= \int^{r}_{0}\int^{r}_{0}\mathrm{d}x\mathrm{d}y\ \frac{1}{2}x\sqrt{r^{2}-x^{2}} + \frac{1}{2}y\sqrt{r^{2} -y^{2}} + xy + \frac{1}{2}r^{2}\left( 2\pi - \arccos\frac{x}{r} - \arccos\frac{y}{r} - \frac{\pi}{2} \right)\nonumber \\
&= \int^{r}_{0}\mathrm{d}x\int^{\sqrt{r^{2}-x^{2}}}_{0}\mathrm{d}y\ \frac{1}{2}x\sqrt{r^{2}-x^{2}} + \frac{1}{2}y\sqrt{r^{2} -y^{2}} + xy + \frac{1}{2}r^{2}\left( 2\pi - \arccos\frac{x}{r} - \arccos\frac{y}{r} - \frac{\pi}{2} \right)\nonumber \\
&= \int^{r}_{0}\mathrm{d}x\left[ \left\{ \frac{1}{2}x\sqrt{r^{2}-x^{2}} + \frac{3}{4}\pi r^{2} - \frac{1}{2}r^{2}\arccos\frac{x}{r} \right\}\sqrt{r^{2}-x^{2}} \right. \nonumber \\
&\ \ \ \ \ \ \ \ \ \ \ \ \ \ + \left.  \int^{\sqrt{r^{2}-x^{2}}}_{0}\mathrm{d}y\ \frac{1}{2}y\sqrt{r^{2}-y^{2}} + xy - \frac{1}{2}r^{2}\arccos\frac{y}{r}\right]\nonumber \\
&= \int^{r}_{0}\mathrm{d}x\left[ \frac{1}{2}xr^{2} - \frac{x^{3}}{2} + \frac{3}{4}\pi r^{2}\sqrt{r^{2}-x^{2}} - \frac{1}{2}r^{2}\arccos \frac{x}{r} \sqrt{r^{2}-x^{2}} \right.\nonumber \\
&\ \ \ \ \ \ \ \ \ \ \ \ \ \ + \left. \frac{r^{3}}{6} - \frac{x^{3}}{6} + \frac{1}{2}xr^{2} - \frac{x^{3}}{2} - \frac{\pi}{4}r^{2}\sqrt{r^{2}-x^{2}} + \frac{1}{2}r^{2}\arccos\frac{x}{r}\sqrt{r^{2}-x^{2}} + \frac{1}{2}r^{2}x - \frac{r^{3}}{2}\right] \nonumber \\
&= \int^{r}_{0}\mathrm{d}x\left[- \frac{7}{6}x^{3} - \frac{r^{3}}{3} + \frac{3}{2}r^{2}x + \frac{\pi}{2}r^{2}\sqrt{r^{2}-x^{2}} \right]\nonumber \\
&= -\frac{7}{24}r^{4} - \frac{r^{4}}{3} + \frac{3}{4}r^{4} + \frac{\pi}{2}r^{2}\cdot \frac{\pi}{4}r^{2}\nonumber \\
&= \left( - \frac{7}{24} - \frac{1}{3} + \frac{3}{4} + \frac{\pi^{2}}{8}\right)r^{4}\nonumber \\
&= \left( \frac{\pi^{2}}{8} + \frac{1}{8} \right)r^{4}
\end{align}
\begin{align}
p'(r)_{\Omega''_{i,j}} &= p'(r)_{\Omega''_{x, y}}\nonumber \\
&= p'(r)_{\Omega''_{x, y}1}  +p'(r)_{\Omega''_{x, y}2}\nonumber \\
&= p'(r)_{\Omega''_{x, y}1'} - p'(r)_{\Omega''_{x, y}1''} + p'(r)_{\Omega''_{x, y}2}\nonumber \\
&= \left(\pi -\frac{4}{3}\right)r^{4} - \frac{\pi^{2}}{8}r^{4} + \left( \frac{\pi^{2}}{8} + \frac{1}{8} \right)r^{4}\nonumber \\
&= \left(\pi -\frac{29}{24}\right)r^{4}
\end{align}

これまでの結果をすべて合わせると、
\begin{align}
p'(r) &= p'(r)_{\Omega''} + 4p'(r)_{\Omega''_{i}} + 4p'(r)_{\Omega''_{i,j}}\nonumber \\
&= (1-2r)^{2}\pi r^{2} + (1-2r)r^{3}\left( 4\pi-\frac{8}{3} \right) + \left(4\pi -\frac{29}{6}\right)r^{4}\nonumber \\
&= \pi r^{2} - 4\pi r^{3} + 4\pi r^{4} + 4\pi r^{3} -\frac{8}{3}r^{3} - 8\pi r^{4} + \frac{16}{3}r^{4} + 4\pi r^{4} - \frac{29}{6}r^{4}\nonumber \\
&= \frac{1}{2}r^{4} -\frac{8}{3}r^{3} + \pi r^{2}
\label{eq:e5}
\end{align}
を得る。

図\ref{fig:f23}に$r$を$0.01$から$0.5$まで変えたとき、領域$\Omega$上に選ばれた1つの点からいくつのエッジが結ばれるか、すなわちその点の次数の期待値を求め、この試行を100回繰り返して平均をとったときのグラフを示す。このときの値はグラフでは青色の曲線であらわされている。一方、先程までの議論の結果として、$\Omega$上にとった1つの点のまわりの$r$による範囲に他の点が存在する確率の期待値は式(\ref{eq:e5})から、
\[p'(r) = \frac{1}{2}r^{4} -\frac{8}{3}r^{3} + \pi r^{2}\]
で表すことができた。よって、一番はじめのモデルで数直線上で考えたときと同じように、次数の期待値は二項分布$B(k,p'(r))$で表すことができるから、$\Omega$全体にある点の個数を$M$とすると、ある$r'$における平均次数は$p'(r')(M-1)$で表せる。グラフ上では緑色の曲線で示された部分がそれである。このときの分散$\sigma^{2}$は、$p'(r')(1-p'(r'))M$である。この分散は1回の試行に関するものであったので、さらに100回の試行を行った今回の偏差$\sigma'$は$\sigma/\sqrt{100}$である。この範囲を示したものが、グラフの中の半透明の緑で表された範囲である。このグラフから、解析的に計算した結果が、実際に測った量とよく一致していることが分かる。
\begin{figure}[H]
    \begin{center}
        \includegraphics[width=12.5cm]{../img/r_l.png}
        \caption{$r$とクラスターの平均次数の関係}
        \label{fig:f23}
    \end{center}
\end{figure}

次に,各ステップ間の平均移動距離$phi$を計算し,それが$N$によってどのように変化するかを図\ref{fig:f24}に示した。このとき$X_{i}$のもつ点の数$S=20$,クラスタ化閾値$r=0.07$とした。
\begin{figure}[H]
    \begin{center}
        \includegraphics[width=10cm]{../img/N_l_2.png}
        \caption{ステップ間距離の平均値$l$と$N$の間の関係}
        \label{fig:f24}
    \end{center}
\end{figure}
このグラフを見て分かるように,$l$は$N$の関数として見たとき下に凸な関数となっている。このようなグラフとなるのは,先程まで考えたように$N$が増えると点の密度が大きくなり,同じ$r$でもクラスターの融合が進むので,結果的にクラスター間の距離は離れることになるということ,それから点の密度が小さいときには,クラスターは形成されにくいが,かわりに各点間の距離は広がるために各ステップ間の距離も大きくなる,と説明できる。

ここまで考えてきた各ステップ間の距離というのは,はじめに設定としてイメージしていた会議を思い浮かべると,意見間の差異を表すことになる。この値が小さいということは,選択された意見の間につながりが見られること,妥当な思考の過程によって次の意見が提出されたことを意味していると見ても良いかもしれない。逆にこの値が大きい時には,意見と意見の間の関連が小さいということを意味しており,それゆえ選択された意見間は,およそつながりがなさそうな意見になっていると言うことができる。つまり突拍子もない意見が提出されている,ということであり,生産的な会議になっているとは考えにくい。




