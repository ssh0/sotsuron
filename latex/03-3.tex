\section{近距離の点をクラスター化するモデル}

このモデルでは、3.2で考えたモデルと同様に、点$x$をそれまでに選択された点からの近さで選んでゆくようなモデルを考えることにする。

これまで考えてきたモデルとの大きな相違は、点を選択していく過程をは始める前に、すべての$x$について、自分から近い位置にある点同士をエッジで結んでいき、クラスターを形成することである。こうして得られたクラスター$y$について、その一つ前に選ばれた点$x$から最も近い点を含む$y$から、次の時刻の点が選ばれるとする。このとき$y$の中から点を選ぶ方法は、そのクラスター内に含まれる点の属するそれぞれの$X$に割り当てられた重みによって決定されるとする。

このモデルの着想を得るにあたって、実際の話し合いの際に、人が多くなることによって人がどのように考えるか、ということに影響を受けている。すなわち、話し合いに参加する人が多いほど、自分の意見と同じ様な意見をもつ人がいるだろう、と思う効果であり、$X_{i}$に割り当てられた重みは、個人の発言力をあらわしていると考えることができる。

シミュレーションにおいてクラスター化を行う際には、自分と他の点との間の距離を測り、これが$r$より小さいものの間にエッジを張ることにする。このとき、(特に$r$が小さいときに)効率よく近傍点を探すことができるよう、はじめに全体を長さ$l(>r)$の正方形の領域(セル)に分割し、それぞれの点がどのセルに属しているかを記録しておく。このようにすると、近傍点を探す際には、自分の属するセルとその周囲8マスを含めた9つのセルの中にある点のみについて調べればよい。このようにしてすべての点について順番に閾値$r$の内部にある点を選択していく。このとき領域内に入ったすべての点にクラスター番号が与えられていない時には、通し番号でクラスター番号を割り当てる。閾値$r$で定められる領域内に、より小さいクラスター番号をもつ点が存在する時には、その中で一番小さいクラスター番号を、接続しているすべての点に同じクラスター番号を付与する。
